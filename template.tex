%==============================常用宏包、环境==============================%
\documentclass[a4]{article}
\usepackage{xeCJK} % For Chinese characters
\usepackage{amsmath, amsthm}
\usepackage[cache=false, outputdir=build]{minted}  % 代码插入与高亮
\usepackage{geometry} % 设置页边距
\usepackage{fontspec}
\usepackage{graphicx}
\usepackage{fancyhdr} % 自定义页眉页脚
\usepackage{hyperref}[hidelinks] % 目录超链接
\setsansfont{Consolas} % 设置英文字体
\setmonofont[Mapping={}]{Consolas} % 英文引号之类的正常显示,相当于设置英文字体
\geometry{left=1cm,right=1cm,top=2cm,bottom=0.5cm} % 页边距
% \setlength{\columnsep}{30pt}
% \setlength\columnseprule{0.4pt} % 分割线
%==============================常用宏包、环境==============================%

%==============================页眉、页脚==============================%
% 页眉、页脚设置
\pagestyle{fancy}
% \lhead{CUMTB}
\lhead{\CJKfamily{hei} Backlight's Code Template}
\chead{}
% \rhead{Page \thepage}
\rhead{\CJKfamily{hei} 第 \thepage 页}
\lfoot{} 
\cfoot{}
\rfoot{}
\renewcommand{\headrulewidth}{0.4pt} 
\renewcommand{\footrulewidth}{0.4pt}
%==============================页眉、页脚==============================%

%==============================标题和目录==============================%
\title{\CJKfamily{hei} \bfseries Backlight's Code Template}
\author{Backlight @ CSU}
\renewcommand{\today}{\number\year 年 \number\month 月 \number\day 日}

\begin{document}\small
\begin{titlepage}
  \maketitle
\end{titlepage}

\newpage
\pagestyle{empty}
\renewcommand{\contentsname}{目录}
\tableofcontents
\newpage\clearpage
\newpage
\pagestyle{fancy}
\setcounter{page}{1}   %new page
%==============================标题和目录==============================%


%==============================正文部分==============================%
\section{ds}
\subsection{SGT}
\inputminted[mathescape,linenos,numbersep=5pt,frame=lines,framesep=2mm]{cpp}{src/ds/SGTree.cpp}
\section{graph}
\subsection{Dijkstra}
\inputminted[mathescape,linenos,numbersep=5pt,frame=lines,framesep=2mm]{cpp}{src/graph/Dijkstra.cpp}
\section{math}
\subsection{Lucas}
\inputminted[mathescape,linenos,numbersep=5pt,frame=lines,framesep=2mm]{cpp}{src/math/Lucas.cpp}
\section{other}
\subsection{BFPRT}
\inputminted[mathescape,linenos,numbersep=5pt,frame=lines,framesep=2mm]{cpp}{src/other/BFPRT.cpp}
\section{string}
\subsection{KMP}
\inputminted[mathescape,linenos,numbersep=5pt,frame=lines,framesep=2mm]{cpp}{src/string/KMP.cpp}

%==============================正文部分==============================%
\end{document}